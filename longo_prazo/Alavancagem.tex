\documentclass[a4paper,12pt]{article}
\usepackage[T1]{fontenc}
\usepackage[utf8]{inputenc}
\usepackage{indentfirst}
\usepackage[brazil]{babel}
\usepackage{lmodern}
\usepackage{amsmath}
% pacote babel ou equivalent

\title{Estrutura de Capital}

\author{Galvão, Ricardo\\
\and 
Galvão, Duzza Gabriella}

\begin{document}

\maketitle
\tableofcontents

\newpage

\begin{abstract}

A temática Alavancagem é um tema relevante não apenas no campo das finanças, mas também nas demais áreas das organizações. Este ensaio apresenta os principais aspectos da alavancagem através de apresentação teórica e exemplos práticos, fazendo parte da série de trabalhos de código aberto do professor Dr. Ricardo Galvão.

\end{abstract}


\section{O que é Alavancagem?}

O significado literal de alavancar é dar impulso a algo, promover o desenvolvimento, aprimorar. No mercado financeiro, o conceito de alavancagem aproxima-se do sentido literal da palavra porque refere-se ao aprimoramento ou à maximização da rentabilidade da empresa com o mínimo de recurso, geralmente por meio do endividamento (uso de ativos ou fundos a custo fixo), que nesse caso atua como uma 'alavanca'. 

Através da estrutura de capital (combinação de capital de terceiros e capital próprio), a empresa reúne recursos que são investidos com o intuito de maximizar o retorno financeiro aos acionistas e credores. Aumentar a alavancagem (endividamento) incorre em maiores retornos e também em maior risco, da mesma forma que diminuí-la reduz o risco e os ganhos.

Há três tipos de alavancagem: operacional, financeira e total, que serão explicadas nas seções seguintes.

\section{Tipos de Alavancagem}

\subsection{Alavancagem Operacional}

A alavancagem operacional tem ligação direta com o aumento das vendas de uma empresa, já que é determinada em função da relação existente entre as receitas operacionais (receita de vendas) e o lucro antes de juros e imposto de renda ou LAJIR (lucro operacional). Neste tipo de alavancagem, a empresa procura aumentar a produção e, consequentemente, as vendas, mantendo fixos os seus custos operacionais.

A medida numérica da alavancagem operacional de uma empresa é o Grau de Alavancagem Operacional (GAO), que calcula a variação do lucro operacional (ou LAJIR) em relação à variação no volume de vendas, podendo ser calculado pela seguinte fórmula:\\

$GAO=\frac{Variação \ percentual \ do \ LAJIR}{Variação \ percentual \ das \ vendas}$\\

Haverá alavancagem operacional quando o GAO for superior a 1, isto é, quando a variação do LAJIR for maior que a variação nas vendas.

É possível perceber que a alavancagem operacional pode ser boa ou ruim para uma empresa, o que vai depender do volume de atividades e de seu crescimento. Se ela obtiver vendas crescentes, a alavancagem operacional resultará em maiores lucros, mas se o volume de vendas cair, a empresa pode ter um prejuízo elevado.


\subsection{Alavancagem Financeira}

A existência do capital de terceiros na estrutura de capital de uma empresa resulta em sua alavancagem financeira. Quando a empresa obtém recursos externos por meio de financiamento, ela assume a responsabilidade pelo pagamento de juros que são despesas dedutíveis do cálculo do imposto de renda, sobrando para os acionistas uma parcela maior de lucro operacional. No entanto, a alavancagem financeira só será favorável se ela obtiver um retorno de suas atividades operacionais maior do que o custo que pagou (juros) pelo capital de terceiros.

Enquanto a alavancagem operacional utiliza custos operacinais fixos, a alavancagem financeira baseia-se na presença de custos financeiros fixos para ampliar a relação entre o LAJIR e o lucro por ação (LPA) de uma empresa. O cálculo do Grau de Alavancagem Financeira (GAF) nos permite avaliar se a alavancagem foi ou não favorável através da fómula:\\

$GAF=\frac{Variação \ percentual \ no \ LPA}{Variação \ percentual \ no \ LAJIR}$\\

Para que a alavancagem financeira seja favorável, o GAF deve ser maior do que 1. Um GAF menor que 1 indica que o custo do capital de terceiros foi maior que o retorno sobre o investimento, trazendo prejuízo ao invés de lucro.


\subsection{Alavancagem Total}

A alavancagem financeira consiste na combinação das alavancagens financeira e operacional, isto é, prioriza, simultaneamente, custos fixos financeiros e operacionais, com a finalidade de ampliar os efeitos da variação nas vendas sobre o Lucro Por Ação (LPA). É importante fizar que quando se usa a alavancagem total ou combinada, o risco também é maior. 

A alavancagem total é calculada por meio do Grau de Alavacangem total (GAT):\\

$GAT= \frac{Variação \ percentual \ no \ LPA}{Variação \ percentual \ nas \ vendas}$\\

Quando o GAT for maior que 1, haverá alavancagem total.

Devido à relação entre as alavancagens financeira e operacional ser multiplicativa, é possível calcular o GAT, o GAO ou o GAF a partir da equação:\\

$GAT= GAO \times GAF$


\section{Casos práticos}

Os casos práticos a seguir foram pensados para exercitar os cálculos dos graus de alavancagem operacional, financeira e total, bem como facilitar sua compreensão em análises de estrutura de capital e maximização de retornos financeiros.

\subsection{Cálculo do Grau de Alavancagem Operacional (GAO)}


\subsection{Cálculo do Grau de Alavancagem Financeira (GAF)}




\subsection{Cálculo do Grau de Alavancagem Total (GAT)}






\end{document}
