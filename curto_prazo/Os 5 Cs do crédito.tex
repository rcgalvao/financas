\documentclass[10pt]{beamer}
\usepackage[brazil]{babel}
\usepackage[utf8]{inputenc}

\usetheme[progressbar=frametitle]{metropolis}
\usepackage{appendixnumberbeamer}

\usepackage{booktabs}
\usepackage[scale=2]{ccicons}

\usepackage{pgfplots}
\usepgfplotslibrary{dateplot}

\usepackage{xspace}
\newcommand{\themename}{\textbf{\textsc{metropolis}}\xspace}

\title{Os cinco C's do crédito}
\subtitle{Gerenciando a concessão de crédito}
%\date{\today}
\date{}
\author{Prof. Dr. Ricardo Galvão}
\institute{www.rgalvao.com}
% \titlegraphic{\hfill\includegraphics[height=1.5cm]{logo.pdf}}

\begin{document}

\maketitle

%\begin{frame}{Conteúdo da apresentação}
%  \setbeamertemplate{section in toc}[sections numbered]
%  \tableofcontents[hideallsubsections]
%\end{frame}

%\section{Introdução}


\begin{frame}[fragile]{Introdução}

A temática cobrança assume papel importante dentro da gestão financeira da empresa uma vez que a inadimplência se converte diretamente em prejuízos.\\
Os cinco C's representam os cinco aspectos mais importantes na concessão de crédito e se resumem em: 
\begin{itemize}
  \item Caráter
  \item Capacidade
  \item Capital
  \item Colateral
  \item Condições 
\end{itemize}
Conhecer os cinco C's do crédito vale a pena por auxiliar o gestor na hora de conceder ou restringir crédito, levando à redução do inadimplemento e aumentando o retorno sobre o capital investido.

\end{frame}


\begin{frame}[fragile]{Caráter}

O caráter representa o histórico da empresa em honrar os próprios compromissos.\\
Empresas com um histórico de pagamento das própria obrigações tendem a continuar honrando os próprios compromissos. Por outro lado, empresas com a prática corriqueira de não cumprimento das obrigações indicam que tal padrão se repetirá.\\
Como conclusão, tem-se que vale a pena verificar a honradez do cliente antes de conceder ou restringir o financiamento das vendas.

\end{frame}


\begin{frame}[fragile]{Capacidade}

Por mais honesto que seja o cliente, há um limite na sua capacidade de honrar as obrigações.\\ Desta forma, vale verificar se a sua geração de caixa já está totalmente comprometida com outras obrigações, sendo necessário restringir a oferta de crédito nesse caso.

\end{frame}


\begin{frame}[fragile]{Capital}

O capital representa o patrimônio líquido do cliente. Empresas que possuem muito capital próprio apresentam risco reduzido porque há muito capital disponível após o pagamento das obrigações, o que facilita o recebimento do montante concedido.

\end{frame}

\begin{frame}[fragile]{Colateral}

O colateral representa as garantias concedidas pelo cliente. \\
A concessão do crédito pode se dar em troca da alienação de ativos tais como estoques, máquinas etc.\\
Desta forma, o limite de crédito concedido pode ser facilitado diante da oferta de ativos como garantia.

\end{frame}

\begin{frame}[fragile]{Condições}

A ideia central é analisar as condições as quais passa a empresa, o setor, o país etc.\\
Momentos de prosperidade facilitam o recebimento do crédito concedido enquanto momentos de crise comprometem o faturamento, a geração de caixa e, consequentemente, a capacidade do cliente honrar as próprias obrigações.

\end{frame}


\begin{frame}[fragile]{Conclusão e considerações}
Esta breve apresentação teve por objetivo compilar de maneira concisa o que são os cinco C's do crédito.\\
Trata-se de mais um documento do projeto \textbf{Repositório de código aberto voltado a textos sobre Finanças} do professor Ricardo Galvão, que tem por objetivo disponibilizar livremente materiais nas áreas de investimentos e finanças seguindo a licença Creative Commons Attribution-ShareAlike.


\end{frame}


\begin{frame}{Créditos}

Tanto esta apresentação quanto as imagens e o tema utilizados estão sob a licença  \href{http://creativecommons.org/licenses/by-sa/4.0/}{Creative Commons
  Attribution-ShareAlike 4.0 International License}.
  \begin{center}\ccbysa\end{center}

A apostila pode ser obtida no endereço:
\begin{center}\url{github.com/rcgalvao/financas}\end{center}

O tema pode ser obtido no endereço: 
\begin{center}\url{github.com/matze/mtheme}\end{center}

\end{frame}


\end{document}
