\documentclass[jou,apacite]{apa6}
\usepackage[brazil]{babel} 
\usepackage[utf8]{inputenc} 
\usepackage{amsmath}
\usepackage{amsfonts}
\usepackage{amssymb}

\title{Custo de agência}
\shorttitle{APA style}

\author{Dr. Ricardo Galvão}
\affiliation{https://github.com/rcgalvao/financas}

\abstract{A temática custo de agência, ou \textit{agency cost}, é um tema relevante não apenas no campo das finanças, mas também nas demais áreas das organizações. Este ensaio apresenta os principais aspectos do custo de agência através de apresentação teórica e exemplos práticos, fazendo parte da série de trabalhos de código aberto do professor Dr. Ricardo Galvão .}

\rightheader{APA style}
\leftheader{Ricardo Galvão}

\begin{document}
\maketitle    
                        
\section{Definição}
Os casos práticos envolvendo problemas de agência são recorrentes. Mesmo sem conhecer a temática ou o mesmo termo \textit{agency cost}, é comum ao brasileiro conhecer tais casos. \\
O custo de agência, ou custo do agente, é precedido pelo termo problema de agência, que será abordado inicialmente, sendo seguido pela apresentação do referido termo custo de agência.\\
Imagine um vendedor com salário fixo (atribuiremos o nome fictício de Carlos)... Obviamente, Carlos foi contratado com o intuito de aumentar o faturamento da empresa através de esforços de vendas. Ao chegarem clientes, haverá um funcionário cujo objetivo é convencer estes clientes de que a compra será vantajosa. Desta forma, a empresa, com a contratação de Carlos, espera que o faturamento aumente de maneira consistente.\\
A ideia é excelente e funcionará perfeitamente se Carlos trabalhar em prol da empresa durante as horas nas quais presta o seu respectivo expediente. Porém, com o passar do tempo, Carlos perceberá que o seu resultado pessoal será o mesmo se o cliente for muito bem atendido e persuadido ou se o atendimento não for bem prestado e o cliente deixar a empresa sem realizar nenhuma compra (lembre-se que Carlos receberá um salário fixo). Aqui apresentamos um aspecto importante: Prestar um atendimento excelente e convencer o cliente de que a compra de determinado produto é a melhor escolha exige mais dedicação e esforço do que tratar com indiferença e sem entusiasmo. Voltando ao nosso exemplo, Carlos poderá diminuir os esforços de venda uma vez que, caso o faça, chegará ao final do dia menos cansado.\\
Se pararmos pra pensar, perceberemos que Carlos possui mais incentivos a não atender bem porque isto exige mais trabalho. Eis o problema de agência: Apesar de ser contratado para se dedicar durante determinadas horas a aumentar o faturamento da empresa através das vendas, Carlos poderá facilmente descumprir o acordo e não se dedicar com afinco a tal atividade. Como conseqência as vendas não atingirão os patamares possíveis.\\
Consideramos como problema de agência (ou de agente, em uma tradução mais adequada) as situações nas quais os indivíduos contratados para atuar em defesa de terceiros (empresas ou pessoas) deixam de cumprir o acordo (mesmo que implícito) e passam a atuar em proveito próprio.\\
Como contornar tal problema enfrentado pela empresa? Pode-se atrelar a remuneração de Carlos ao volume de vendas por ele conseguido. Esta é a famosa comissão de vendas. Com a adoção da remuneração variável em função do faturamento, Carlos ganhará mais coforme venha a cumprir com o seu papel dentro da empresa. Com isso, ficam melhor alinhados os interesses de Carlos e da empresa. Sendo assim, a empresa reduz parte da sua magem de lucro (aumentam os gastos em virtude do pagamento de comissões), mas provavelmente ganhará bem mais em virtude do aumento no faturamento.\\
Ao gasto despendido para melhorar este alinhamento entre interesses do agente e do contratante damos o nome de custo de agência.\\
Problemas de agência são mais comuns do que se imagina. Políticos não costumam utilizar os serviços públicos que gerenciam (normalmente utilizam serviços privados de saúde, educação e segurança) e podem agir atendendo a interesses distintos dos quais foram eleitos. Situação semelhante ocorre em algumas repartições públicas nas quais aos servidores não é dado incentivo a prestar um bom atendimento. Mesmo em empresas privadas é comum o desalinhamento entre interesses de funcionários e empresas. Porém, os efeitos dos problemas de agência mais estudados estão na remuneração de executivos. Quando o presidente de uma empresa não é o proprietário desta, as decisões por ele tomadas podem alterar completamente o desempenho futuro e a criação de valor para os proprietários.\\
Como você pode perceber, este tema é decisivo na execução da estratégia das organizações e seu estudo rende boas recompensas em termos de resultados. A seguir apresentaremos os conceitos, seguidos pela apresentação de alguns trabalhos teóricos que abordaram a temática, depois apresentaremos exemplos que facilitarão a compreensão. Por fim, haverá uma breve conclusão com a síntese de conceitos.


\section{Problema de agência}
Problema de agência ocorre quando um indivíduo contratado para representar o interesse de uma organização age em favor de outrem. Desta forma, configura-se como problema de agência o fato (ou a possibilidade) de um dos agentes não atuar em favor daquele que o contratou.  \\
A título de exemplo, se um administrador é contratado para gerenciar uma organização, espera-se que aja em favor dos acionistas ao criar valor para esta organização. O problema de agência ocorre quando este passa a tomar decisões em proveito próprio ou até em favor de outras organizações, descumprindo o acordo de trabalho e destruindo valor na organização. São exemplos comuns os casos de problemas de agência no serviço público uma vez que o gestor, normalmente eleito, vai gerenciar recursos de terceiros sem que haja a figura de um proprietário daqueles ativos sob controle dele. 

\section{Custos de agência}

\section{Trabalhos teóricos}


\section{Casos práticos}


\section{Conclusão}





Results are presented in Table~\ref{tab1}.
\begin{table}[!htb]
\caption{Sample table.}\label{tab1}
\begin{tabular}{ccc}
\hline\\[-1.5ex]
AAA & BBB & CCC \\[0.5ex]
\hline\\[-1.5ex]
1.0 & 2.0 & 3.0\\[0.5ex]
1.0 & 2.0 & 3.0\\[0.5ex]
\hline
\end{tabular}
\end{table}


\bibliography{bibliografia}

\end{document}

