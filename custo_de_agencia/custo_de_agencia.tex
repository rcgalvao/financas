\documentclass[jou,apacite]{apa6}
\usepackage[brazil]{babel} 
\usepackage[utf8]{inputenc} 
\usepackage{amsmath}
\usepackage{amsfonts}
\usepackage{amssymb}

\title{Custo de agência}
\shorttitle{APA style}

\author{Dr. Ricardo Galvão}
\affiliation{https://github.com/rcgalvao/financas}

\abstract{A temática custo de agência, ou \textit{agency cost}, é um tema relevante não apenas no campo das finanças, mas também nas demais áreas das organizações. Este ensaio apresenta os principais aspectos do custo de agência através de apresentação teórica e exemplos práticos, fazendo parte da série de trabalhos de código aberto do professor Dr. Ricardo Galvão .}

\rightheader{APA style}
\leftheader{Ricardo Galvão}

\begin{document}
\maketitle    
                        
\section{Definição}
Os casos práticos envolvendo problemas de agência são recorrentes. Mesmo sem conhecer a temática ou o mesmo termo \textit{agency cost}, é comum ao brasileiro conhecer tais casos. \\
O custo de agência, ou custo do agente, é precedido pelo termo problema de agência, que será abordado inicialmente, sendo seguido pela apresentação do referido termo custo de agência.\\


\section{Problema de agência}
Problema de agência ocorre quando um indivíduo contratado para representar o interesse de uma organização age em favor de outrem. Desta forma, configura-se como problema de agência o fato (ou a possibilidade) de um dos agentes não atuar em favor daquele que o contratou.  \\
A título de exemplo, se um administrador é contratado para gerenciar uma organização, espera-se que aja em favor dos acionistas ao criar valor para esta organização. O problema de agência ocorre quando este passa a tomar decisões em proveito próprio ou até em favor de outras organizações, descumprindo o acordo de trabalho e destruindo valor na organização. São exemplos comuns os casos de problemas de agência no serviço público uma vez que o gestor, normalmente eleito, vai gerenciar recursos de terceiros sem que haja a figura de um proprietário daqueles ativos sob controle dele. 

\section{Custos de agência}

\section{casos teóricos}

\section{Casos práticos}


\section{Conclusão}





Results are presented in Table~\ref{tab1}.
\begin{table}[!htb]
\caption{Sample table.}\label{tab1}
\begin{tabular}{ccc}
\hline\\[-1.5ex]
AAA & BBB & CCC \\[0.5ex]
\hline\\[-1.5ex]
1.0 & 2.0 & 3.0\\[0.5ex]
1.0 & 2.0 & 3.0\\[0.5ex]
\hline
\end{tabular}
\end{table}


\bibliography{bibliografia}

\end{document}

